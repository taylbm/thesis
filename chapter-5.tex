\chapter{Chapter Five}
\section{Conclusions}
The Greater Golden Horseshoe region in southern Ontario is highly susceptible to lake-effect snow. C-Band radar is the current tool used to observe and nowcast
these events in real-time. This tool will soon be replaced by S-Band radar, with its wider beamwidth and higher elevation angles. A case study comparing lake
effect snow events, with synoptic snow events used a control, has been undertaken to assess the relative merits of these two types of radar systems. With the
data transformed to a common grid, the base variables of two neighboring radars in this region are compared. These two radars are Environment and Climate
Change Canada's King City C-Band radar and the National Weather Service's Buffalo, NY S-Band radar. In terms of $Z_{eH}$, subset comparisons indicate that the
higher elevation angle of the S-Band radar could be lead to overshooting, and underestimation of the strength of the snow-squall. This problem is especially 
acute at mid to long ranges. For $Z_{DR}$, S-Band radar shows advantages over C-Band in comparatively observing a larger range of values. The confidence in 
these findings are enhanced by estimating joint probability density functions of matched variables. It is shown that this method can reduce noise and improve 
the quality of the data. It essentially distills the massive amount of information which radars provide to the areas of meteorological interest.