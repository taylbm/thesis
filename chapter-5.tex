\chapter{Chapter Five}
\section{Conclusions}
The Greater Golden Horseshoe region in Southern Ontario is highly susceptible to lake-effect snow. C-Band radar is the current tool used to observe and nowcast
these events in real-time. This tool will soon be replaced by S-Band radar, with its wider beamwidth and higher elevation angles similiar to KBUF. A case study comparing lake
effect snow events, with synoptic snow events used a control, has been undertaken to assess the relative merits of these two types of radar systems. With the
data transformed to a common grid, the base variables of two neighboring radars in this region are compared. These two radars are Environment and Climate
Change Canada's King City C-Band radar and the National Weather Service's Buffalo, NY S-Band radar. Comparisons indicate that these radars are calibrated well relative to one another, as the bias-adjusted values for both $Z_{eH}$ and $Z_{DR}$ are similiar. In terms of $Z_{eH}$, subset comparisons indicate that the higher elevation angle of the S-Band radar leads to overshooting, and slight underestimation of the strength of the snow-squall.  This problem is especially 
acute at mid to long ranges. The difference is small enough that it does not limit forecasters ability to nowcast for these events. For $Z_{DR}$, S-Band radar shows advantages over C-Band in comparatively observing a larger range of values due to a larger beam volume. In terms of hydrometeor types observed in lake-effect snow versus synoptic snow, aggregated snow is more prevalent in lake-effect events. Also, multi-modal distributions of hydrometeors appear more often in more intense snowfall events. These findings are enhanced by estimating joint probability density functions of matched variables. It is shown that this method can reduce noise and improve 
the quality of the data, by removing erroneous data caused by partial beam blockages, low SNR, temporal and spatial mismatching, and even transmit and receive errors. The KDE constrainment method essentially distills the massive amount of information which radars provide to the most valuable areas for meteorological interests.