\chapter{Chapter Two}
\section{Methodology}
\subsection{Comparison of Radar Systems}
Comparing two radar datasets is fraught with challenges; solutions to meet this challenge are presented herein. Even though the radar system characteristics
are not identical, the measurements are comparable due to the design of the weather radar equation, which accounts for the sensitivity of the radar system
itself \citep{Rogers1989}. The area of study was chosen to ensure that the coinciding radar scans had similiar resolution samples and beam heights. 
\begin{figure}[h]
\includegraphics[]{study_area}\centering
\caption{Bounding box of the study area, denoted by the green shading.} 
\label{fig:study_area}
\end{figure}
Lake Ontario happens to be the perfect area to bound between the radars, therefore only data from areas over water inside the bounding box depicted in Figure
\ref{fig:study_area} are used. This also ensures that no ground clutter is incorporated into the analyses.
\subsubsection{Comparing Radar Characteristics}

As presented in Equation \ref{eq:weather_eq}, the weather radar equation is defined by constant parameters dependent on the radar system characteristics, and
varying properties related to the target. This equations determines the theoretical mean power ($\bar{P_r}$) that should be returned to the radar receiver.
\begin{equation}\label{eq:weather_eq}
\bar{P_r} = \frac{\pi^3c}{1024 \ln(2)} \Bigg[\frac{P_t \tau G^2 \theta^2}{\lambda^2}\Bigg]_{dBZ_0} \Bigg[\mathcolorbox{yellow} {||K_{w}||^{2}}\frac{Z_{eh}}{r^2}\Bigg]_{TARGET}
\end{equation}
The target properties are dielectric constant (K), range (r) and equivalent reflectivity factor $Z_{eH}$. Conversely, the radar parameters ideally remain
unchanged from their values upon installation of the radar system. These parameters form the radar constant, symbolically expressed as $dBZ_0$. The
parameters that define this constant include the power transmitted ($P_t$), the pulse length ($\tau$), the antenna gain (G), the angular beamwidth
($\theta$), and the wavelength ($\lambda$).
\begin{equation}\label{eq:radpow}
\mathcolorbox{yellow}{Z_{eH}} = 10 \log{\bar{P_r}} + 20 \log r - dBZ_0
\end{equation}
Equation \ref{eq:radpow} shows how $dBZ_0$ is subtracted out from the full calculation of $Z$. This means that values from two different systems are comparable, as the contributions to the returned power intrinsic to the radar are negated. Next, Table \ref{radarspecs} compares parameters for both
radar systems. The biggest difference between the two is the wavelength, with CWKR operating in the C frequency band and KBUF operating in the S frequency
band. It should be noted that although KBUF has a larger physical beamwidth than CWKR, it achieves an effective azimuthal resolution of $0.5^{\circ}$ through
an over-sampled data windowing technique \citep{Torres2007}. Therefore, the two radars are matched in azimuthal resolution, while CWKR has twice the range
resolution of KBUF. Also, it should be stated that the signal processors used in both radar systems are in the
Vaisala SIGMET series, therefore they measure
$Z_{eH}$ and $Z_{DR}$ using 8 bit resolution. For $Z_{eH}$, the data intervals of -31.5 dBZ to +95.5 dBZ yield a
data resolution of 0.5 dBZ.  For $Z_{DR}$, an interval of -7.94 dB to +7.94 dB yields a resolution of 0.0625 dB.

\begin{table}[h]
    \caption{Specifications of each radars system, with symbols as used in Eq. \ref{eq:radpow}. Pulse Length is specified as short pulse/long pulse. \hl{CWRN is an acronym for Canadian Weather Radar Network.}}\label{radarspecs}
    \begin{center}
    \begin{tabular}{|l|c|c|c|}
    \hline
     field [symbol](unit) & King City (CWKR) & Buffalo (KBUF) & \hl{New CWRN} \\
    \hline\hline
    Wavelength [$\lambda$](cm) & 5.6  & 10.7 & 10.5\\
    \hline
    Beamwidth [$\theta$] ($^\circ$)  & 0.62  & 0.92 & 0.88 \\
    \hline
     Antenna Gain [G] (dB) & 45.5 & 49.2 & 45.8\\
    \hline
     Peak Power (kW) & 250 & 1000 & 1000 \\
    \hline
     Pulse Length [$\tau$] ($\mu s$) &  0.8/2.0 & 1.5/4.5 & 0.4/4.5 \\
    \hline
     Lowest Elevation Angle ($^\circ$) & 0.2 & 0.5 & 0.4 \\
    \hline
     Range Resolution [$r$] (m)& 125 & 250 & 500 \\
    \hline
    \end{tabular}
    \end{center}
\end{table}


\subsection{Distance-Weighting Scheme}
The biggest challenge when comparing radar resolution volumes measured by radars that are not co-located is resolving the differences in coordinate system. A
resolution volume is defined as volume irradiated by the idealized Gaussian beam pattern for each range gate, otherwise known as a bin. Resolution volumes
are sampled natively in the spherical coordinate system; although there may be some overlap, the shape of the bins will vary drastically. Differences between
KBUF and CWKR bin geometry can be ascertained from Figure \ref{fig:compare_bins}. 
These differences require the radar data to be objectively analyzed onto a common coordinate system, which can be achieved through a distance-weighting
scheme. This method was adopted in the open source software module called the Python ARM Radar Toolkit (Py-ART) \citep{Py-ART}, which is used here. In
accordance with the recommendations of \cite{Pauly1990}, a grid resolution ($\Delta x$, $\Delta y$) of 500 meters is chosen.A Barnes distance-weighting scheme is used for this analysis. 
\begin{equation}\label{eq:barnesdws}
f^{'}_{p} = \sum_{b=1}^n  e^{-(d_b/\kappa)^{2}} f_q  \bigg/ \sum_{b=1}^n e^{-(d_b/\kappa)^{2}}
\end{equation}
The Gaussian weighting function used in said scheme is given in Equation \ref{eq:barnesdws}. It shows that the value of the analysis at some point $p$ is
equal to sum of weights convolved with the actual values at bin $b$ in radar space, divided by the sum of the weights. The summation is performed over $n$
number of bins that are within the radius of influence ($\kappa$) of the center point of the grid cell, and $d_{b}$ is is the horizontal distance from the native
bin to the center point of the cell. Vertical distance is neglected, as only the lowest elevation angle from the radars are included for comparison.
\begin{equation}\label{eq:roi}
\kappa = d_{0} * \tan{\theta}
\end{equation}
The definition of $\kappa$ is found in Equation \ref{eq:roi}, where $d_{0}$ is the horizontal distance from the grid cell to the radar and $\theta$ is the angular\
beamwidth. This completes the framework for comparing the radar datasets in this study.
\begin{figure}[t!]
\centering
   \begin{subfigure}{1.0\textwidth} \centering
     \includegraphics[width=1\linewidth]{compare_bins_ref}
     \caption{$Z_H$ comparison, shows the transformation from a smooth input to smooth gridded output.}\label{fig:compare_ref}
   \end{subfigure}
   \begin{subfigure}{1.0\textwidth} \centering
     \includegraphics[width=1\linewidth]{compare_bins_zdr}
     \caption{$Z_{DR}$ comparison, in contrast with (a), shows the limitations of representing an non-linear field with an isotropic distance-weighting
     function. }\label{fig:compare_zdr}
   \end{subfigure}
\caption{Base moment comparisons between radars over Lake Ontario, with dimensions of 20x12.5 km. Left panels are in native radars coordinates, with gates
outlined in black. Right panels are transformed to a common Cartesian grid, with grid cells outlined in black.} \label{fig:compare_bins}
\end{figure}

\section{Selection of Cases}
\hl{Five lake-effect snow events are chosen as the experimental dataset, while five synoptic snow events are chosen as a control dataset. Synoptic snowfall are a good control for this experiment, as they typically span greater vertical distances than shallow lake-effect effect events. In this way, neither radar has an advantage in sampling the low-level features characteristic of lake-effect snow bands.}
Cases selected for this study were chosen entirely based on the pattern of motion and banding of the radar
echoes. Radar mosaics for the study area were
manually examined, beginning in 2014. When time intervals with echoes in the study area were observed, it was
noted whether they moved repeatedly over the same area,
or were progressive.  Also, it was noted whether they exibited meso-$\beta$ length scales ($\sim$ 20 km wide),
or synoptic length scales ($>$ 200 km) \citep{Orlanski1975}. The narrow bands that moved over the same area
were classified as lake-effect driven events, while the progressive, wide-areas of precipitation are classified as
synoptically driven events. A tabulation of important level temperatures for the five lake-effect snow events selected
is shown in Table \ref{eventslake}, along with the time interval during which radar scans were chosen. Synoptic
events are given in Table \ref{synopticevents}. This indicates that all events were sufficiently below freezing, and
dry snow/ice crystals were the predominant hydrometeor type.

\begin{table}[H]
    \caption{\hl{Temperatures at the top of the cloud and surface from radiosonde launched closest in time to the selected lake-effect snow events, with the time interval during which radar scans were chosen. Top T is the temperature at the top of the cloud determined by using the radiosonde level closest to the mean max echo top height. Mean max echo top is tabulated as Echo Tops.}}\label{eventslake}
    \begin{center}
    \begin{tabular}{|l|c|c|c|c|c|}
    \hline
     BUF - Radiosonde & Radar Times & \hl{Echo Tops} & Top T ($^{\circ}C$) & Sfc. T ($^{\circ}C$)\\
    \hline\hline
    2014-01-23 00Z & 0100-1000Z & 1.9 km & -21.6 & -14.9 \\
    \hline
    2015-01-06 12Z & 1200-1700Z & 0.5 km & -15.3 & -11.7 \\
    \hline
    2015-02-14 12Z & 1000-1400Z & 2.1 km & -21.0 & -6.9 \\
    \hline
    2015-02-18 12Z & 2100-2359Z & 1.9 km & -21.1 & -10.1 \\
    \hline
    2016-02-10 12Z & 1300-2359Z & 1.9 km & -14.0 & -2.7 \\
    \hline
    \end{tabular}
    \end{center}
\end{table}

\begin{table}[H]
    \caption{\hl{Temperatures at the top of the cloud and surface from radiosonde launched closest in time to the selected synoptic snow events, with the time interval during which radar scans were chosen. Top T is the temperature at the top of the cloud determined by using the radiosonde level closest to the mean max echo top height. Mean max echo top is tabulated as Echo Tops.}}\label{synopticevents}
    \begin{center}
    \begin{tabular}{|l|c|c|c|c|c|}
    \hline
     KBUF - Radiosonde & Radar Times & \hl{Echo Tops} & Top T ($^{\circ}C$) & Sfc. T ($^{\circ}C$)\\
    \hline\hline
    2014-01-18 12Z & 0600-0800Z & 2.4 km & -15.7 & -6.5 \\
    \hline
    2014-02-01 12Z & 1500-1800Z & 4.3 km & -15.0 & -3.1 \\
    \hline
    2015-01-07 12Z & 0900-1100Z & 3.2 km & -37.5 & -10.1 \\
    \hline
    2015-02-06 12Z & 0900-1030Z & 3.9 km & -25.9 & -10.7 \\
    \hline
    2016-12-15 12Z & 0920-1020Z & 3.5 km & -39.7 & -12.3 \\
    \hline
    \end{tabular}
    \end{center}
\end{table}
\section{Filtering Conditions}
Several conditions were used to narrow down the selected sets to the best suited scans and individual gates for admission into the distance-weighting scheme.
\subsection{Time Filter}
Scan start times are compared between the radars, and if they are within four minutes of each other, the pair is admitted. For CWKR, there is a regular
volume update frequency of ten minutes, while KBUF is variable based on the Volume Coverage Pattern (VCP) selected by the operator. The update frequency
could be as short as every two minutes if the operator has activated Supplemental Adapative Intra-Volume Low-Level Scans (SAILS) mode.
\subsection{Gate Filters}
Several gate filters were use to ensure the highest data quality possible. Gates with \hl{$\rho_{hv} < 0.95$} were excluded to filter for dry snow and crystals
only. A manual inspection of all cases was made, and the range of $Z_{DR}$ in all cases lay within $-0.5 < Z_{DR} < 1.5$ dB. Therefore, $Z_{DR}$ values
outside this range were filtered to avoid contamination from beam blockages, etc. Finally, gates over land were filtered, to avoid ground clutter
contamination.
\section{Advanced Statistical Techniques}
Scatter plots directly comparing grid cells produced by the distance-weighting scheme are used in this study. This section discusses how advanced statistical
techniques were leveraged to derive the most information from these plots, and also reduce the error intrinsic to the variables for quantitative analysis.
\subsection{Bi-Variate Kernel Density Estimates}
Both radar datasets contain a similiar amount of measurement and analysis error. Furthermore, scatter-plots containing on the order of $10^5$ points become
overwhelming to visually analyze. To solve this problem, a bi-variate \hl{Kernel Density Estimate (KDE) is calculated. This calculation estimates} the joint probability density function between two random variables \citep{Silverman1986}. The two variables compared in this study are the matched
observations made by the two radars. Equation \ref{eq:kde_full} gives a full definition of this estimate, where $\mathbf{x}$ is the matrix of matched
observations, $\mathbf{H}$ is the 2x2 bandwidth (smoothing) matrix, and $\mathbf{K}$ is the kernel function. \citet{Scott1992} suggests a rule of thumb for
calculating the bandwidth matrix, shown in Equation \ref{eq:scotts_rule}, where $n$ is the number of points and $\sigma$ is the standard deviation. The
kernel function is chosen as 2-D Gaussian throughout, given as Equation \ref{eq:gauss_kern}, where the terms retain their prevailing meaning. Figure
\ref{fig:KDE_example} demonstrates the motivation and the discrete version of this method, a 2-D histogram of the data. The units of the KDE can be thought
of as a likelihood ratio.
\\
\begin{equation}\label{eq:kde_full}
\hat{f}_{\mathbf{H}}(\mathbf{x}) = \frac{1}{n} \sum_{i=1}^{n} \mathbf{K_H}(\mathbf{x} - \mathbf{x}_i)
\end{equation}
\begin{equation}\label{eq:scotts_rule}
\mathbf{H} = \sigma * n^{-1/3}
\end{equation}
\begin{equation}\label{eq:gauss_kern}
\mathbf{K_H}(\textbf{x}) = \mathbf{H} * 2\pi * e^{-\frac{1}{2} \mathbf{x^T} \mathbf{H^{-1}} \mathbf{x}}
\end{equation}
\begin{figure}
\includegraphics[width=\textwidth]{KDE_example}\centering
\caption{Scatter-plot of matched points illustrating the high density of the points (left). The total of points in this figure are 147,457. 2-D histogram of
the same matched points, binned to native data resolution (right).} 
\label{fig:KDE_example}
\end{figure}

\subsection{Orthonormal Linear Regression}
A hallmark of this study is the lack of ground truth. The sample sets compared contain error prone, independent variables. Typically, scatter-plots compare
an independent variable to a dependent variable. Instead of performing a standard linear regression between the variables, an orthonormal linear regression
is used. This type of regression allows for error in both variables, by performing the least squares regression perpendicular to the initial fit instead of
vertically \citep{Markovsky2007}. Figure \ref{fig:total_least_squares} demonstrates this concept.
\begin{figure}[H]
\includegraphics[scale=0.4]{total_least_square}\centering
\caption{Demonstration of an Orthonormal Linear Regression} 
\label{fig:total_least_squares}
\end{figure}
\section{$Z_{DR}$ Hardware Bias Estimation}
Although it not possible to check absolute calibration of $Z_{H}$ when comparing two radars, it is possible to verfiy $Z_{DR}$ calibration due to relative
nature of the quantity \citep{Zrnic2006}. While radars are regularly calibrated using internal calibration procedures, an external check is useful for
monitoring the time-varying component of calibration. \hl{This time-varying component arises due to hardware issues, i.e. differences between the split pathways that the horizontal and vertical channels take through the transmit/receive chain.} The typical process for calibration of $Z_{DR}$ is pointing the antenna to zenith and performing ``bird
bath'' scans during light rain events \citep{Hubbert2006}, \hl{so called due to the antenna resemblance to a bird-bath when performing these scans}. The $Z_{DR}$ in light rain is expected to be 0 dB, therefore any offset from this is considered a
bias; The signal processor subtracts out this bias to achieve the final output. Due to mechanical constraints, NEXRAD radars are unable to perform this
procedure, but CWKR is. NEXRAD radars disseminate a product which contains an estimate of $Z_{DR}$ bias using the intrinsic properties of
dry snow \cite{Zittel2015}. The daily bias reported in the NEXRAD Archived Status Product will be used to adjust $Z_{DR}$ values obtained from KBUF to diagnose any bias at CWKR. Under normal conditions, $Z_{DR}$ can be calibrated within 0.1 dB \citep{Zrnic2006}. This error threshold is adopted in this study.