The Canadian Weather Radar Network is currently undergoing an upgrade to polarimetric, S-Band radar 
systems. Forecasting experiences in Canada with the legacy C-Band radars lends to the idea that the 
narrow beamwidth of C-Band systems is preferential for nowcasting the typical shallow lake-effect snow 
event. This idea is tested by comparing moments from King City radar, just north of Toronto, to the 
neighboring Buffalo, NY WSR-88D. By transforming the radar data from spherical coordinates to the 
Cartesian coordinate system, the two radars can be compared directly. Objective analysis indicates that 
the spatial patterns of reflectivity are very similiar, with King maintaining the obvious advantage in 
resolving fine scale features of lake-effect snow bands through a narrow physical beamwidth. Also, it is 
shown that comparatively, the mean reflectivity values obtained through this method are within 1 dB. In
regards to differential reflectivity, a case by case comparison is performed to determine any event biases
from the King City radar. This is achieved by subtracting out the bias of KBUF; this value comes from 
estimates made using the NEXRAD external target bias estimation techniques. It is shown that 
differential reflectivities at King City can become positively biased with snow on the radome, with an average bias near +0.25 dB. Without snow on the radome, the differential reflectivity 
bias as measured in comparison with KBUF remains within the error threshold of $\pm$0.1 dB.