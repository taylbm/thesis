The Canadian Weather Radar Network is currently undergoing an upgrade to po-
larimetric, S-Band radar systems. Forecasting experiences in Canada with the
legacy C-Band radars lends to the idea that the narrow beamwidth of C-Band systems
is preferential for nowcasting the typical shallow lake-effect snow event. This
idea is tested by comparing moments from King City radar, just north of Toronto,
to the neighboring Buffalo, NY WSR-88D. By transforming the radar data from
spherical coordinates to the Cartesian coordinate system, the two radars can be
compared directly. Objective analysis indicates that the spatial patterns of reflec-
tivity are very similiar, with King maintaining the obvious advantage in resolving
fine scale features of lake-effect snow bands through a narrow physical beamwidth.
Also, it is shown that comparatively, the mean reflectivity values obtained through
this method are similiar, but King City maintains a slight advantage over Buffalo in
detecting shallow snow-squalls. In regards to differential reflectivity, a case by case
comparison is performed to determine any event biases from the King City radar. With
biases removed, both radars indicate similiar mean values of differential reflectivity, which agrees with
theoretical expectations. Results also indicate that the bulk hydrometeor type in synoptic snowfalls tend towards
pristine crystals, while lake-effect events tend towards aggregated snow.