Calibrating weather radars is an ongoing issue, and as radar networks age and requirements change, new methods will continue to be developed to address these issues. When the United States WSR-88D NEXRAD weather radar network was upgraded to dual-polarimetric capabilities from 2011-2013, there was a flurry of effort to handle the calibration requirements of the new equipment. Techniques for the estimating bias of differential reflectivity ($Z_{DR}$) using external targets were developed to verify that the internal calibration procedures performed as expected. With the calibration of one radar known, it can be compared with another to verify the others performance. In this case, the National Weather Service Buffalo, NY WSR-88D is compared with its neighbor to the north, Environment Canada's King City radar. Comparisons are performed during two different subsets of precipitation events, those being synoptic and lake-effect snow events. The data are analyzed onto a common grid using a distance-weighting scheme, with a hydrometeor classification scheme used to filter for dry snow. It is shown that the agreement between the radars in terms of reflectivity is within the bounds of the canonical 1 dB bias threshold. Furthermore, while the previous  external target method relies on the self-consistency principle and is only able to detect a negative bias, this method brings in an independent set of observations to diagnose both positive and negative biases. 